\documentclass[runningheads,a4paper]{llncs}


\usepackage[utf8]{inputenc}
\usepackage{graphicx}
\usepackage{listings, color}
\usepackage[bookmarks,bookmarksopen,bookmarksdepth=2]{hyperref}


\usepackage{amsmath}
\usepackage{amsfonts}
\usepackage{amssymb}
\usepackage{mathrsfs}


\title{An Mereological and Algebraic Formalization of MegaL's Prelude}
\author{Maximilian Meffert}
%\institute{Universtiy of Koblenz-Landau}
\date{\today}


\newtheorem{axiom}{Axiom}
%\newtheorem{definition}{Axiom}[section]


\newcommand{\ENTITY}{\mathsf{ENTITY}}
\newcommand{\RELATIONSHIP}{\mathsf{RELATIONSHIP}}
\newcommand{\MEGA}{\mathsf{MEGA}}



\begin{document}

\maketitle

{
\Large
\noindent 
\textbf{Disclaimer:}
\newline\noindent
These notes are at draft level and may be very imperfect.
\newline\noindent
Copyright lies with the author.
}

\begin{abstract}
...
\end{abstract}

\begin{keywords}
megamodeling, 
metamodeling, 
mereology, 
mereotopology,
parthood, 
conformance,
correspondence,
representation
\end{keywords}

\section{Introduction}
...

\section{A Mereological Formalization}

\subsection{Parthood}


A mereology focuses on the \textbf{parthood} predicate $\mathcal{P}$.  
We read the $\mathcal{P}xy$ as \textit{"x is part of y"} for two arbitrary objects $x$ and $y$.
\cite{DBLP:journals/dke/Varzi96}


\begin{definition}[Parthood]
Parthood is a binary predicate $\mathcal{P}$ between two entities x and y. We write
\begin{align}
\mathcal{P} x y
\end{align}
and read \textit{"x is part of y"}.
\end{definition}

\begin{axiom}[Parthood is Reflexive]
\begin{align}
\mathcal{P}xx
\end{align}
An object x is always part of itself.
\end{axiom}
Note that this opens the door for problems similar to Russel's antinomy for naive sets. 
\begin{axiom}[Parthood is Antisymmetric]
\begin{align}
\mathcal{P}xy \wedge \mathcal{P}yx \Rightarrow x=y
\end{align}
If two objects are part of one another, we assume both objects to be identical.
\end{axiom}
\begin{axiom}[Parthood is Transitive]
\begin{align}
\mathcal{P}xy \wedge \mathcal{P}yz \Rightarrow \mathcal{P}xz
\end{align}
We assume parthood to be transitive in the sense of object composition.
If one object is part of  a second object, which in turn is part of a third object, the first is considered to be also a part of the latter one.
\end{axiom}

\subsubsection{Specializations of Parthood}
We define the following specializations of parthood:

\begin{definition}[Proper Part]
\begin{align}
\mathcal{PP}xy := \mathcal{P}xy \wedge \neg \mathcal{P}yx
\end{align}
\end{definition}

\begin{definition}[Overlap]
\begin{align}
\mathcal{O}xy := \exists z (\mathcal{P}zx \wedge \mathcal{P}zy)
\end{align}
\end{definition}

\begin{definition}[Underlap]
\begin{align}
\mathcal{U}xy := \exists z (\mathcal{P}xz \wedge \mathcal{P}yz)
\end{align}
\end{definition}


\subsection{Correspondence}


Correspondence in the sense that two objects are similar to one another.


\begin{definition}[Correspondence]
Correspondence is a binary predicate $\mathcal{K}$ between two entities x and y. We write
\begin{align}
\mathcal{K} x y
\end{align}
and read \textit{"x corresponds to y"}.
\end{definition}

\begin{axiom}[Correspondence is Reflexive]
\begin{align}
\mathcal{K} x x
\end{align}
\end{axiom}


\begin{axiom}[Correspondence is Symmetric]
\begin{align}
\mathcal{K} x y \Rightarrow \mathcal{K} y x
\end{align}
\end{axiom}

\begin{axiom}[Correspondence and Parthood]
\begin{align}
\mathcal{K} e_1 e_2 
\Rightarrow \exists p_1,p_2(\mathcal{P} p_1 e_1 \wedge \mathcal{P} p_2 e_2 \wedge \mathcal{K} p_1 p_2)
\end{align}
\end{axiom}

\subsection{Conformance}

\begin{definition}[Conformance]
Conformance is a binary predicate $\mathcal{C}$ between two entities x and y. We write
\begin{align}
\mathcal{C} x y
\end{align}
and read \textit{"x conforms to y"}.
\end{definition}


\begin{axiom}[Conformance and Parthood]
\begin{align}
\mathcal{C} x y \Rightarrow \forall u \exists v (\mathcal{P}uy \wedge \mathcal{P}vx \Rightarrow \mathcal{C} u v) 
\end{align}
\end{axiom}


\subsection{Representation}


\begin{definition}[Representation]
Representation is a binary predicate $\mathcal{K}$ between two entities x and y. We write
\begin{align}
\mathcal{R} x y
\end{align}
and read \textit{"x is a representation of y"}.
\end{definition}

\newpage

\section{An Algebraic Formalization}

\begin{definition}[Entities]
\begin{align}
\ENTITY := \text{"Set of all entities"}
\end{align}
\end{definition}

\begin{definition}[Relationships]
\begin{align}
\RELATIONSHIP := \ENTITY \times \ENTITY
\end{align}
Set of all binary relationships between entities
\end{definition}

\begin{definition}[Megamodel]
A megamodel is a four-tuple
\begin{align}
M := (ET, E, RT, R)
\end{align}
with:
\begin{itemize}

\item A set of \textbf{Entity-Types}
\begin{align}
ET \subseteq \{ X \subseteq \ENTITY \}
\end{align}

\item A set of \textbf{Entities} (also called instances)
\begin{align}
E \subseteq \ENTITY
\end{align}

\item A set of \textbf{Relationship-Types}
\begin{align}
RT \subseteq \{ X \subseteq \RELATIONSHIP \}
\end{align}

\item A set of \textbf{Relationships}
\begin{align}
R \subseteq \RELATIONSHIP
\end{align}

\end{itemize}
\end{definition}

\begin{definition}[The Empty Megamodel]
\begin{align}
0 := (\emptyset,\emptyset,\emptyset,\emptyset)
\end{align}
\end{definition}

\begin{definition}[The Set of All Megamodels]
\begin{align}
\MEGA := \{ (ET,E,RT,R) \}
\end{align}
\end{definition}

\begin{definition}[Megamodel Inclusion]
\begin{align}
\begin{split}
& + : \MEGA \times \MEGA \rightarrow \MEGA
\\ & M_1 + M_2 := (ET_1 \cup ET_2, E_1 \cup E_2, RT_1 \cup RT_2, R_1 \cup R_2)
\end{split}
\end{align}
\end{definition}

\begin{proposition}[Megamodel Inclusion is Commutative]
\label{prop:mega-inc-com}
\begin{align}
\forall M_1,M_2 \in \MEGA : M_1 + M_2 = M_2 + M_1 
\end{align}
\textbf{Proof:}
\begin{equation}
\begin{split}
& M_1 + M_2 
\\ & = (ET_1 \cup ET_2, E_1 \cup E_2, RT_1 \cup RT_2, R_1 \cup R_2)
\\ & = (ET_2 \cup ET_1, E_2 \cup E_1, RT_2 \cup RT_1, R_2 \cup R_1)
\\ & = M_2 + M_1
\end{split}
\end{equation}
\end{proposition}

\begin{proposition}[Megamodel Inclusion is Associative]
\label{prop:mega-inc-ass}
\begin{align}
\forall M_1,M_2,M_3 \in \MEGA : M_1 + (M_2 + M_3) = (M_1 + M_2) + M_3
\end{align}
\textbf{Proof:}
\begin{equation}
\begin{split}
& M_1 + (M_2 + M_3)
\\ & = (ET_1,E_1,RT_1,R_1) + ((ET_2,E_2,RT_2,R_2) + (ET_3,E_3,RT_3,R_3))
\\ & = (ET_1,E_1,RT_1,R_1) + (ET_2 \cup ET_3, E_2 \cup E_3, RT_2 \cup RT_3, R_2 \cup R_3)
\\ & = (ET_1 \cup (ET_2 \cup ET_3), E_1 \cup (E_2 \cup E_3),RT_1 \cup (RT_2 \cup RT_3), R_1 \cup (R_2 \cup R_3))
\\ & = ((ET_1 \cup ET_2) \cup ET_3, (E_1 \cup E_2) \cup E_3,(RT_1 \cup RT_2) \cup RT_3, (R_1 \cup R_2) \cup R_3)
\\ & = (ET_1 \cup ET_2, E_1 \cup E_2, RT_1 \cup RT_2, R_1 \cup R_2) + (ET_3,E_3,RT_3,R_3)
\\ & = ((ET_1,E_1,RT_1,R_1) + (ET_2,E_2,RT_2,R_2)) + (ET_3,E_3,RT_3,R_3)
\\ & = (M_1 + M_2) + M_3
\end{split}
\end{equation}
\end{proposition}

\begin{proposition}[Megamodel Inclusion Identity]
\label{prop:mega-inc-zero}
\begin{align}
\forall M \in \MEGA : M + 0 = 0 + M = M
\end{align}
\textbf{Proof:}
\begin{equation}
\begin{split}
M + 0
& = (ET,E,RT,R) + (\emptyset,\emptyset,\emptyset,\emptyset)
\\ & = (ET \cup \emptyset,E \cup \emptyset,RT \cup \emptyset,R \cup \emptyset)
\\ & = (ET,E,RT,R) = M
\end{split}
\end{equation}
$M + 0 = 0 + M$ follows directly from proposition \ref{prop:mega-inc-com}.
\end{proposition}

\begin{proposition}[Megamodel Inclusion Semigroup]
\label{prop:mega-inc-semigroup}
\begin{align}
(\MEGA,+) \text{ is a Semigroup}
\end{align}
\textbf{Proof:}
Follows directly from proposition \ref{prop:mega-inc-ass}.
\end{proposition}


\begin{proposition}[Megamodel Inclusion Abelian Semigroup]
\label{prop:mega-inc-abelian-semigroup}
\begin{align}
(\MEGA,+) \text{ is an Abelian Semigroup}
\end{align}
\textbf{Proof:}
Follows directly from propositions \ref{prop:mega-inc-semigroup} and \ref{prop:mega-inc-com}.
\end{proposition}

\begin{proposition}[Megamodel Inclusion Abelian Monoid]
\label{prop:mega-inc-abelian-monoid}
\begin{align}
(\MEGA,+,0) \text{ is an Abelian Monoid}
\end{align}
\textbf{Proof:}
Follows directly from propositions \ref{prop:mega-inc-abelian-semigroup} and \ref{prop:mega-inc-zero}.
\end{proposition}


\begin{definition}[Megamodel]
A megamodel is a labeled directed multigraph
\begin{align}
M := (\Sigma_V, \Sigma_A, V, A, s, t, l_V, l_A)
\end{align}
with
\begin{itemize}

\item A set of vertices $V = \mathscr{E} \cup \mathscr{M}$
\item A set of arcs $A = V \times V$
\item Two alphabets $\Sigma_V$ and $\Sigma_A$


\end{itemize}
\end{definition}


\section{Other Stuff}

\subsubsection{A Mereology for Naive Sets}
Let $A,B$ be sets.
\begin{axiom}[Element Parthood]
\begin{align}
a \in A \Rightarrow \mathcal{P} a A
\end{align}
\end{axiom}

\begin{axiom}[Subset Parthood]
\begin{align}
A \subseteq B \Rightarrow \mathcal{P} A B
\end{align}
\end{axiom}

\begin{axiom}[Proper Subset Parthood]
\begin{align}
A \subset B \Rightarrow \mathcal{PP} A B
\end{align}
\end{axiom}


\subsubsection{A Mereology for Tuples}
Let $p = (a,b) \in A \times B$

\subsubsection{A Mereology for Formal Languages}

\subsection{The Parthood Relationship}

\begin{align}
\begin{split}
& (.)  \subset \RELATIONSHIP
\\ & (.) := \{ (e_1,e_2) \in \RELATIONSHIP ~|~ \mathcal{P} e_1 e_2 \}
\end{split}
\end{align}

\begin{align}
\forall e \in \ENTITY : e . e
\end{align}

\begin{align}
\forall x,y \in \ENTITY : x . y \wedge y . x \Rightarrow x = y
\end{align}

\begin{align}
\forall x,y,z \in \ENTITY : x . y \wedge y . z \Rightarrow x . z
\end{align}




\bibliographystyle{splncs}
\bibliography{template_Article}{}

\end{document}
